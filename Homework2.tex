   \documentclass[12pt]{article}
\usepackage[utf8]{inputenc}
\usepackage{amsmath,amsthm}
\usepackage{fullpage}
\usepackage{amsfonts}
\usepackage{amssymb,multicol}
\usepackage{enumitem}
\usepackage[colorlinks=true]{hyperref}
\usepackage{xcolor}

\newcommand{\Z}{\mathbb{Z}}
\newtheorem*{theorem}{Theorem}

\newlist{checklist}{itemize}{2}
\setlist[checklist]{label=$\square$}

\begin{document}
\begin{center}
{\Large Homework 2}\\
Due: Friday, September 23 at 11:59 pm CDT\\


\end{center}
{\bf Instructions:} Submit a pdf of your solutions to the HW 2 assignment on Gradescope. 

When working on this assignment, you should focus on the following goals:
\begin{itemize}
\item Carefully use and cite the bijection principle.
\item Carefully use and cite the multiplication principle.
%\item Use the definitions of combinations and permutations to solve counting problems.
\item Write clear and correct proofs that meet the \href{https://docs.google.com/document/d/18LfQoqi6BsY2VdAlpC5xdYEA2rxSGoH0891nVec4_Os/edit?usp=sharing}{Specifications for a Well-written Solution} with a special focus on the following conventions of a mathematical proof:
\begin{checklist}
\item All conventions emphasized in Homework 1.
\item Tell your reader what you are proving and what you are assuming. 
\item Use transition words to show the logical connections between your sentences.
\end{checklist}
\end{itemize}


\begin{enumerate}
\item Choose {\bf one} of the following two exercises.
\begin{enumerate}
\item Establish a bijection between the set of dominoes consisting of double blank through double $n$ and the set of edges in the complete graph on $n+2$ vertices. 
\item Complete Problem 10 (including a closed form for a general n).  (Hint: to get from the recurrence to a closed form, you might want to look back at the solution for the tower of Hanoi in chapter 0)
\end{enumerate}
\item Complete problem 7 from the Supplementary exercises for Chapter 1 section of the appendix.  (You should explain your reasoning in complete sentences, but I am not expecting a novel 1-3 sentences to explain where each number comes from in your calculations is sufficient.)
\item Complete problem 11 from the Supplementary exercises for Chapter 1 section of the appendix.
\item Assuming $k\le n$, in how many ways can we pass out 
$k$ distinct pieces of candy (distinct means that each piece of candy is different from the others) to $n$ children if each child may get at most one piece? What is the number if $k>n$? Assume for both questions that we pass out all the candy. (Answer this question using only the product principle. In other words, do not use a formula for counting combinations or premutations.)

\end{enumerate}

\end{document}