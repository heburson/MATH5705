   \documentclass[12pt]{article}
\usepackage[utf8]{inputenc}
\usepackage{amsmath,amsthm}
\usepackage{fullpage}
\usepackage{amsfonts}
\usepackage{amssymb,multicol}
\usepackage{enumitem}
\usepackage[colorlinks=true]{hyperref}
\usepackage{xcolor}

\newcommand{\Z}{\mathbb{Z}}
\newtheorem*{theorem}{Theorem}

\newlist{checklist}{itemize}{2}
\setlist[checklist]{label=$\square$}

\begin{document}
\begin{center}
{\Large Midterm Assignment}\\
Due: Sunday, October 30\\


\end{center}
{\bf Instructions:} Submit a pdf of your responses to the Midterm assignment on Gradescope.  For parts 4 and 5, you can think of what you are submitting as rough drafts so I can give you feedback and confirm that you are on the right track for the final portfolio. 


\begin{enumerate}
\item Write 5 true/false questions that illustrate a variety of ideas from this course that you might put on this exam if you were teaching the class. Give a key, explain the answers, then explain why you chose these particular questions and what you hope they will assess. \\
To earn credit, your questions should satisfy the following criteria:
\begin{itemize}
\item At least two statements must be false.
\item The questions must be different from questions that have appeared on class activities, homework, or exams. (However, it is fine for some of your questions to come from making small changes to statements/questions from class or assignments).
\item For each question, you must include the answer, an explanation of why that is the answer, and an explanation of why you chose that question and what you hope it assesses. 
\end{itemize} 
\item One of the course goals for this class is to learn to communicate mathematics orally. Now that you have presented several times, what are your strengths when it comes to oral communication of mathematics? What is something you want to get better at?  What is a realistic change you will make to help acheive that goal? How can your classmates and Dr.~Burson help you acheive your goal?
\item Pick one homework problem or daily preparation problem that you did not have a correct solution for by the deadline. Explain where you got stuck or what was missing in your solution and what you have learned since then. Then, write a revised version of your solution.  For full credit, this revised version should meet all of the homework specifications. 
\item (Optional, but highly recommended) Look at the section of the syllabus about your final portfolio. Choose one component from components 2-5 and write a rough draft. 
\item  (Optional, but highly recommended) Look at the techniques that are listed in component 1 of the final portfolio. For any of the techniques you have used so far, find one or two 'S'-marked homework problems. Write down the assignment number and problem number and explain how you used the technique. 

\end{enumerate}
\end{document}