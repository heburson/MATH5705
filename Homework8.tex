   \documentclass[12pt]{article}
\usepackage[utf8]{inputenc}
\usepackage{amsmath,amsthm}
\usepackage{fullpage}
\usepackage{amsfonts}
\usepackage{amssymb,multicol}
\usepackage{enumitem}
\usepackage[colorlinks=true]{hyperref}
\usepackage{xcolor}

\newcommand{\Z}{\mathbb{Z}}
\newtheorem*{theorem}{Theorem}

\newlist{checklist}{itemize}{2}
\setlist[checklist]{label=$\square$}

\begin{document}
\begin{center}
{\Large Homework 8}\\
Due: Friday, December 9\\
\end{center}
{\bf Instructions:} Submit a pdf of your solutions to the HW 8 assignment on Gradescope. 

Make sure that your solutions meet the \href{https://docs.google.com/document/d/18LfQoqi6BsY2VdAlpC5xdYEA2rxSGoH0891nVec4_Os/edit?usp=sharing}{Specifications for a Well-written Solution} 

\begin{enumerate}
\item Let $K$ be a set with $k$ elements. In this problem, you will determine the number of surjective functions $K$ to the set $[n]$.  For each of the questions below, carefully explain your answer and reasoning. 
\begin{enumerate}
\item First, we will consider all functions $f:K\to [n]$ (not just surjections). For each $i\in [n]$, define the set $A_i:=\{f:K\to [n]\mid f(x)\ne i \text{ for every } x\in K\}$.  If $f$ is a surjective (onto) function, how many of these $A_i$s does $f$ belong to? 
\item Let $I\subseteq [n]$ such that $|I|=j$. What is $|\bigcap_{i\in I} A_i|$?
\item Using your work from parts (a) and (b), use the Inclusion-Exclusion Principle to determine the number of surjections from $K$ to $[n]$. 
\end{enumerate} 

\item Without determining a closed-form formula for the nth term of the sequence, find the initial terms of and a homogeneous linear recurrence equation satisfied by the sequence whose generating function is 
$$\frac{3}{1-9z+3z^2}.$$ Make sure to fully and carefully explain your answer. 

\item Using generating functions, prove that, for $d\ge 2$, the number of partitions of $n$ with no parts divisible by $d$ is equal to the number of partitions of $n$ where no parts appears $d$ or more times. 
\item Using a bijection, prove that, for $d\ge 2$, the number of partitions of $n$ with no parts divisible by $d$ is equal to the number of partitions of $n$ where no parts appears $d$ or more times. 


\item Prove that the number of standard Young Tableaux of shape $n+n$ is equal to the $n$th Catalan number. 
\end{enumerate}
\end{document}
